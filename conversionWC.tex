\documentclass[letterpaper,11pt]{article}
\pdfoutput=1
%\hyphenpenalty=500
%\widowpenalty=1000
%\clubpenalty=500

% packages
%\usepackage{jheppub}
\usepackage{mathtools,amssymb,braket,mathrsfs,tikz,subfig,xspace,xcolor,float,color,siunitx,comment,afterpage,multirow,array,hhline,amsthm,framed}
\usepackage[utf8]{inputenc}
\usepackage[countmax]{subfloat}

% bib style
%\bibliographystyle{JHEP}

%\usepackage{graphicx,rotating,amsmath,amssymb,xcolor,upgreek,cancel,placeins,url}[h]
\usepackage{fullpage}
\usepackage{afterpage}
\usepackage{pdfpages}
\usepackage{color}
\usepackage{amsfonts}
\usepackage{blindtext}
\usepackage{hyperref}
\usepackage{cite}

\title{EFT Wilson Coefficent Conversions}
\date{}
\author{Lukas Lechner, Robert Sch\"ofbeck}

\begin{document}
\maketitle
\noindent

\tableofcontents
\clearpage
\section{Conversions in \texttt{dim6top}}
\subsection{C$_\text{tB}$ to C$_\text{tZ}$ conversion}
\label{sec:ctb}
\noindent
Basis transformation of the Wilson coefficient C$_\text{tB}$ to C$_\text{tZ}$ using the Weinberg angle~$\theta_W$ and $\sin\theta_W~\approx~0.471$
\begin{align}
	C_\text{tZ}  &= \textrm{Re}\left( -\sin\theta_{W} C_\text{tB} + \cos\theta_{W} C_\text{tW}\right),
\end{align}

\noindent
where only one Wilson coefficient at a time is used, thus for the conversion of C$_\text{tB}$ to C$_\text{tZ}$, the coefficent C$_\text{tW}$ is set to 0.
\begin{align}
	C_\text{tZ} = \textrm{Re}\left(-\sin\theta_{W}\,C_\text{tB}\right) \approx -0.471\,C_\text{tB}
\end{align}

\noindent
\fbox{%
    \parbox{\textwidth}{%
		\centering \textbf{C$_\text{tB}$ limits translate to C$_\text{tZ}$ limits with a factor of $\approx$ -0.471}
    }%
}

\subsection{C$_{\phi\text{Q}}$ conversion}
\label{sec:cpQM}
\noindent
In the \texttt{FeynRules} model \texttt{dim6top}, a linear combination of Wilson coefficients is used to parametrize the Z boson coupling. The coefficent \texttt{cpQM} is given as
\begin{align}
	C^-_{\phi Q} = C_{\phi q}^{1(33)} - C_{\phi q}^{3(33)},
\end{align}

\noindent
where additionally \texttt{cpQ3} is defined as $C_{\phi q}^{3(33)}$.
If \texttt{cpQM} is the only Wilson coefficient set to non-zero, its definition reduces to
\begin{align}
	C^-_{\phi Q} = C_{\phi q}^{1(33)}
\end{align}

\noindent
which removes effects of \texttt{cpQM} on the tWb vertex, but adds effects on the bbZ coupling.
This can be seen in the operator definitions of the processes

\begin{align}
	O_{\phi q}^{1(ij)} &= (\phi^\dag i\overleftrightarrow{D_\mu}\phi)(\bar{q}_i\gamma^\mu q_j)\\
	O_{\phi q}^{3(ij)} &= (\phi^\dag  i\overleftrightarrow{D_\mu}^I\phi)(\bar{q}_i\gamma^\mu \tau^I q_j)
\end{align}

\noindent
where $\tau^I$ are the Pauli matrices and thus, $O_{\phi q}^{3(ij)}$ introduces anomalous couplings to the W boson.
\clearpage

\section{Conversion of ATLAS Wilson coefficients}
ATLAS presented ttZ limits on EFT in the talk of Ref.~\cite{ATLAS:Top2018}.
However different definitions of the operators~$\mathcal{O}$ are used in ATLAS~\cite{Bylund:2016phk} compared to CMS~\cite{AguilarSaavedra:2018nen}, leading to a scale factor for the Wilson coefficients.

\subsection{C$_\text{tB}$ conversion: ATLAS vs. CMS}
\label{sec:ATLAS_CMS_tB}

To convert the definitions, the Lagrangians have to be equal.
\begin{align}
	C_\text{tB}^\text{ATLAS}\,\mathcal{O}^\text{ATLAS}_\text{tB} &\stackrel{!}{=} C_\text{tB}^\text{CMS}\,\mathcal{O}^\text{CMS}_\text{tB}\label{eq:op_eq}\\
	\mathcal{O}^\text{CMS}_\text{tB} &= \left(\bar{Q}\sigma^{\mu\nu}t\right)\tilde{\phi}B_{\mu\nu}\\
	\mathcal{O}^\text{ATLAS}_\text{tB} &= y_t\,g_Y\,\left(\bar{Q}\sigma^{\mu\nu}t\right)\tilde{\phi}B_{\mu\nu} = y_t\,g_Y\,\mathcal{O}^\text{CMS}_\text{tB}.\label{eq:op_Atlas_CMS}
\end{align}

\noindent
Using Eq.~\ref{eq:op_eq} and Eq.~\ref{eq:op_Atlas_CMS} leads to
\begin{align}
	C_\text{tB}^\text{CMS} &= y_t\,g_Y C_\text{tB}^\text{ATLAS}
\end{align}

\noindent
where $y_t=\frac{\sqrt{2}m_t}{v}$. Using the definitions from Ref.~\cite{Bylund:2016phk}
\begin{align*}
	m_t &= 173.3~\text{GeV}\\
	\alpha_\text{EW} &= \frac{1}{127.9}\\
	v &= 246~\text{GeV}\\
	\tan\theta_W &= \frac{g_1}{g_2} = 0.535\\
	\alpha &= \frac{1}{4\pi}\frac{(g_1g_2)^2}{g_1^2+g_2^2} = \frac{g_1^2}{4\pi(1+\tan^2\theta_W)} = \frac{g_1^2\cos^2\theta_W}{4\pi} = \frac{1}{127.9}\\
	g' &= g_1 = g_Y = \sqrt{4\pi\alpha(1+\tan^2\theta_W)} = \frac{\sqrt{4\pi\alpha}}{\cos\theta_W} \approx 0.34\\
	g &= g_2 = g_w = \frac{g_1}{\tan\theta_W} \approx 0.636\\
	y_t &= \frac{\sqrt{2}m_t}{v} \approx 0.996
\end{align*}

\noindent
leads to a conversion factor for $C_\text{tB}$ from ATLAS to CMS of
\begin{align}
C_\text{tB}^\text{CMS} &= y_t\,g_Y C_\text{tB}^\text{ATLAS} \approx 0.34 \, C_\text{tB}^\text{ATLAS}
\end{align}

\noindent
\fbox{%
    \parbox{\textwidth}{%
		\centering \textbf{The ATLAS C$_\text{tB}$ limits translates to limits used in CMS with a factor of $\approx$ 0.34}
    }%
}

\subsubsection{Total conversion from C$_\text{tB}^\text{ATLAS}$ to C$_\text{tZ}^\text{CMS}$ limits}

To convert from C$_\text{tB}^\text{ATLAS}$ to C$_\text{tZ}^\text{CMS}$, an additional conversion factor for C$_\text{tB}$ to C$_\text{tZ}$ from Sec.~\ref{sec:ctb} has to be applied.
\begin{align}
	C_\text{tZ}^\text{CMS} = -\sin\theta_{W}\,y_t\,g_Y C_\text{tB}^\text{ATLAS} \approx -0.16\,C_\text{tB}^\text{ATLAS}
\end{align}

\noindent
\fbox{%
    \parbox{\textwidth}{%
		\centering \textbf{The ATLAS C$_\text{tB}$ limits translates to CMS C$_\text{tZ}$ limits with a factor of $\approx$ -0.16}
    }%
}

\subsubsection{Conversion of ATLAS ttZ C$_\text{tB}^\text{ATLAS}$ limits}

The presented ttZ ATLAS C$_\text{tB}^\text{ATLAS}$ limits of Ref.~\cite{ATLAS:Top2018} convert to CMS C$_\text{tZ}^\text{CMS}$ limits as
\begin{align}
	C_\text{tZ}^\text{ATLAS$\rightarrow$ CMS}(36.1~\text{fb}^{-1}, 95\%, \text{x-sec only}):\, [-2.40,+2.40]
\end{align}


\subsection{C$_\text{tW}$ conversion: ATLAS vs. CMS}
\label{sec:ATLAS_CMS_tW}

To convert the definitions, the Lagrangians have to be equal.
\begin{align}
	C_\text{tW}^\text{ATLAS}\,\mathcal{O}^\text{ATLAS}_\text{tW} &\stackrel{!}{=} C_\text{tW}^\text{CMS}\,\mathcal{O}^\text{CMS}_\text{tW}\label{eq:op_eqW}\\
	\mathcal{O}^\text{CMS}_\text{tW} &= (\bar{q}_i\sigma^{\mu\nu}\tau^Iu_j)\tilde{\phi}W_{\mu\nu}^I\\
	\mathcal{O}^\text{ATLAS}_\text{tW} &= y_t\,g_w\,(\bar{q}_i\sigma^{\mu\nu}\tau^Iu_j)\tilde{\phi}W_{\mu\nu}^I = y_t\,g_w\,\mathcal{O}^\text{CMS}_\text{tW}\label{eq:op_Atlas_CMSW}
\end{align}

\noindent
Using Eq.~\ref{eq:op_eqW} and Eq.~\ref{eq:op_Atlas_CMSW} leads to
\begin{align}
	C_\text{tW}^\text{CMS} &= y_t\,g_w C_\text{tW}^\text{ATLAS} \approx 0.633 \, C_\text{tW}^\text{ATLAS}
\end{align}

\noindent
where $y_t=\frac{\sqrt{2}m_t}{v}\approx 0.996$ and $g_w \approx 0.636$ (see Sec.~\ref{sec:ATLAS_CMS_tB}).\\

\noindent
\fbox{%
    \parbox{\textwidth}{%
		\centering \textbf{The ATLAS C$_\text{tW}$ limits translates to limits used in CMS with a factor of $\approx$ 0.633}
    }%
}

\subsection{C$_{\phi\text{t}}$ and C$_{\phi\text{Q}}$ conversion: ATLAS vs. CMS}
\label{sec:ATLAS_CMS_pQ}

%There is no conversion factor for C$_{\phi\text{t}}$, C$_{\phi\text{Q}}^\text{(1)}$, C$_{\phi\text{Q}}^\text{(3)}$ and C$_{\phi\text{Q}}$.\\

%\noindent
%\fbox{%
%    \parbox{\textwidth}{%
%		\centering \textbf{The ATLAS C$_{\phi\text{t}}$, C$_{\phi\text{Q}}^\text{(1)}$, C$_{\phi\text{Q}}^\text{(3)}$ and C$_{\phi\text{Q}}$ limits translate to limits used in CMS with a factor of 1.0.}
%    }%
%}

To convert the definitions, the Lagrangians have to be equal.
\begin{align}
	C_{\phi\text{t}}^\text{ATLAS}\,\mathcal{O}^\text{ATLAS}_{\phi\text{t}} &\stackrel{!}{=} C_{\phi\text{t}}^\text{CMS}\,\mathcal{O}^\text{CMS}_{\phi\text{t}}\label{eq:op_eqPhi}\\
	\mathcal{O}^\text{CMS}_{\phi\text{t}} &= (\phi^\dag i\overleftrightarrow{D_\mu}\phi)(\bar{t}\gamma^\mu t)\\
	\mathcal{O}^\text{ATLAS}_{\phi\text{t}} &= \frac{1}{2}y_t^2\,(\phi^\dag i\overleftrightarrow{D_\mu}\phi)(\bar{t}\gamma^\mu t) = \frac{1}{2}y_t^2\,\mathcal{O}^\text{CMS}_{\phi\text{t}}\label{eq:op_Atlas_CMSPhi}
\end{align}

\noindent
However, the factor $\frac{1}{2}$ cancels, as the definition of ATLAS~\cite{Bylund:2016phk} includes the hermitian conjugate state in the Lagrangian, where the CMS definitions~\cite{AguilarSaavedra:2018nen} uses the fact that these operators are hermitian.
Thus, using Eq.~\ref{eq:op_eqPhi} and Eq.~\ref{eq:op_Atlas_CMSPhi} leads to
\begin{align}
	C_{\phi\text{t}}^\text{CMS} &= y_t^2\,C_{\phi\text{t}}^\text{ATLAS} \approx 0.992\,C_{\phi\text{t}}^\text{ATLAS}
\end{align}

\noindent
where $y_t=\frac{\sqrt{2}m_t}{v}\approx 0.996$ (see Sec.~\ref{sec:ATLAS_CMS_tB}).
The same pre-factor is applied to C$_{\phi\text{Q}}^{(1)}$ and C$_{\phi\text{Q}}^{(3)}$, thus C$_{\phi\text{Q}}$.\\

\noindent
\fbox{%
    \parbox{\textwidth}{%
		\centering \textbf{The ATLAS C$_{\phi\text{t}}$, C$_{\phi\text{Q}}^\text{(1)}$, C$_{\phi\text{Q}}^\text{(3)}$ and C$_{\phi\text{Q}}$ limits translate to limits used in CMS with a factor of 0.992.}
    }%
}



\clearpage

\section{Conversion of HEL UFO Wilson coefficients}
Comparison of the Wilson coefficients using couplings of Lagrangians from the \texttt{couplings.py} files in the \texttt{FeynRules} directories for the \texttt{dim6top-} and \texttt{HEL-UFO} models. 
Here, only an example of one coupling for one Lagrangian is used to show the conversion factor, where the Lagrangians are labeled similar to the definitions in the \texttt{couplings.py} file (GC \#).


\subsection{C$_\text{tB}$ to C$_\text{tZ}$ conversion}

\begin{align}
	g(\mathcal{L}_\text{dim6}, \text{GC 664}) &= \frac{C_\text{tZ}}{\Lambda^2\sqrt{2}}\\
	g(\mathcal{L}_\text{HEL}, \text{GC 2191}) &= \frac{\sqrt{4\pi\alpha_\text{EW}}\,y_t}{M_W^2\sqrt{2}} \left(C_\text{uB}\frac{\sin\theta_W}{\cos\theta_W} - C_\text{uW}\frac{\cos\theta_W}{\sin\theta_W}\right)\\
	g(\mathcal{L}_\text{dim6}) &\stackrel{!}{=} g(\mathcal{L}_\text{HEL})
\end{align}

\noindent
but only one Wilson coefficient at a time is used, thus C$_\text{uW}=0$, leading to a direct comparison of $C_\text{tZ}$ and $C_\text{uB}$ (see Sec.~\ref{sec:ctb}):
\begin{align}
	\frac{C_\text{tZ}}{\Lambda^2} = -C_\text{uB}\frac{\sqrt{4\pi\alpha_\text{EW}}\,y_t}{M_W^2}\frac{\sin\theta_W}{\cos\theta_W}
	%C_\text{tZ} = C_\text{uB}\frac{\sqrt{4\pi\alpha_\text{EW}}\,y_t\,\Lambda^2}{M_W^2}\frac{\sin\theta_W}{\cos\theta_W}
\end{align}

\noindent
A second example is the comparison of the Lagrangians $g(\mathcal{L}_\text{dim6}, \text{GC 948})$ and $g(\mathcal{L}_\text{HEL}, \text{GC 2187})$, leading to the same results.
Even though it may look reasonable to compare the Lagrangian in switched order (DIM6: GC 664 $\leftrightarrow$ HEL: GC 2187, DIM6: GC 948 $\leftrightarrow$ HEL: GC 2187), the results are then not consistant.

\subsubsection{Comparison of Limits in Top 17-005}

In Ref.~\cite{Sirunyan:2017uzs}, a factor~$k$ is applied to the presented limits, where $k$ is taken from Ref.~\cite{annawoodardgithub}. %Furthermore, the limits are presented in the form $\bar{C}_\text{uB} / \Lambda^2\,[\text{TeV}^{-2}]$
\begin{align}
k &= \frac{\sqrt{4\pi\alpha_\text{EW}}\,y_t}{2\,M_W^2\,\cos\theta_W}\\
\bar{C}_\text{uB} &= C_\text{uB}\,k = C_\text{uB}\,\frac{\sqrt{4\pi\alpha_\text{EW}}\,y_t}{2\,M_W^2\,\cos\theta_W}
%\bar{C}_\text{uB} / \Lambda^2 &= C_\text{uB}\frac{k}{\Lambda^2} = C_\text{uB}\,\frac{\sqrt{4\pi\alpha_\text{EW}}\,y_t}{2\,M_W^2\,\cos\theta_W\,\Lambda^2}
\end{align}

\noindent
To compare the limits to current efforts using the \texttt{dim6top} model, the conversion factor is thus

\begin{align}
\rightarrow \frac{C_\text{tZ}}{\Lambda^2} = -2\,\sin\theta_W\,\bar{C}_\text{uB} = -0.96\,\bar{C}_\text{uB}
%\rightarrow \frac{C_\text{tZ}}{\Lambda^2}\,[\text{TeV}^{-2}] &= C_\text{tZ}\,[(\Lambda/\text{TeV})^{2}] = -2\,\sin\theta_W\,\bar{C}_\text{uB}\,[\text{TeV}^{-2}]
\end{align}

\noindent
\fbox{%
    \parbox{\textwidth}{%
		\centering \textbf{The HEL UFO $\bar{\text{C}}_\text{uB}$ limits in Ref.~\cite{Sirunyan:2017uzs} translate to limits of C$_\text{tZ}$ with a factor of -0.96}\\
    }%
}
%\textbf{The HEL UFO $\bar{\text{C}}_\text{uB}$ limits in Ref.~\cite{Sirunyan:2017uzs} translate to limits of C$_\text{uB}$ with a factor of 2}

\subsubsection{Conversion of Top 17-005 $\bar{\text{C}}_\text{uB}$ limits}

Thus, the presented ttZ $\bar{\text{C}}_\text{uB}$ limits of Ref.~\cite{Sirunyan:2017uzs} convert to \texttt{dim6top} limits as
\begin{align}
\text{C}_\text{tZ}^\text{Top 17-005$\rightarrow$ dim6top}(36.1~\text{fb}^{-1}, 95\%, \text{x-sec only}):\, [-2.02,+2.02]
\end{align}

\subsection{C$_\text{Hu}$ to C$_\text{$\phi$t}$ conversion}

\begin{align}
	g(\mathcal{L}_\text{dim6}, \text{GC 1112}) &= -i\frac{C_{\phi\text{t}}\,\cos\theta_W\,\sqrt{4\,\alpha_\text{EW}\pi}\,v}{\Lambda^2\,\sin\theta_W} - i\frac{C_{\phi\text{t}}\,\sqrt{4\,\alpha_\text{EW}\pi}\,\sin\theta_W\,v}{\cos\theta_W\,\Lambda^2}\\
	g(\mathcal{L}_\text{HEL}, \text{GC 1136}) &= -i\frac{C_\text{Hu}\,\cos\theta_W\,\sqrt{4\,\alpha_\text{EW}\pi}}{\sin\theta_W\,v} - i\frac{C_\text{Hu}\,\sqrt{4\,\alpha_\text{EW}\pi}\,\sin\theta_W}{\cos\theta_W\,v}\\
	g(\mathcal{L}_\text{dim6}) &\stackrel{!}{=} g(\mathcal{L}_\text{HEL})\\
	\rightarrow \frac{C_{\phi\text{t}}}{\Lambda^2} &= \frac{C_\text{Hu}}{v^2}
	%\rightarrow \frac{1}{2}\,\underbrace{\frac{C_{\phi\text{t}}\,v^2}{\Lambda^2}}_\text{limits on \texttt{cpt}} &= \underbrace{\frac{C_\text{Hu}}{v^2}}_\text{limits on \texttt{cHu}}
\end{align}


\subsubsection{Comparison of Limits in Top 17-005}

In Ref.~\cite{Sirunyan:2017uzs}, a factor~$k$ is applied to the presented limits, where $k$ is taken from Ref.~\cite{annawoodardgithub}.  %Furthermore, the limits are presented in the form $\bar{C}_\text{Hu} / \Lambda^2\,[\text{TeV}^{-2}]$
\begin{align}
k &= \frac{1}{2\,v^2}\\
\bar{C}_\text{Hu} &= C_\text{Hu}\,k = \frac{C_\text{Hu}}{2\,v^2}
%\bar{C}_\text{Hu} / \Lambda^2 &= \frac{C_\text{Hu}}{k\,\Lambda^2} = C_\text{Hu}\frac{2\,v^2}{\Lambda^2}
\end{align}

\noindent
To compare the limits to current efforts using the \texttt{dim6top} model, the conversion factor is thus

\begin{align}
\rightarrow \frac{C_{\phi\text{t}}}{\Lambda^2} &= 2\,\bar{C}_\text{Hu}
%\rightarrow \frac{C_{\phi\text{t}}}{\Lambda^2}\,[\text{TeV}^{-2}] &= C_{\phi\text{t}}\,[(\Lambda/\text{TeV})^{2}] = 2\,\bar{C}_\text{Hu}\,[(\Lambda/\text{TeV})^{2}] = 2\frac{\bar{C}_\text{Hu}}{\Lambda^2}\,[\text{TeV}^{-2}]
\end{align}

\noindent
\fbox{%
    \parbox{\textwidth}{%
		\centering \textbf{The HEL UFO $\bar{\text{C}}_\text{Hu}$ limits in Ref.~\cite{Sirunyan:2017uzs} translate to limits of C$_{\phi\text{t}}$ with a factor of 2.0}\\
    }%
}

\subsubsection{Conversion of Top 17-005 $\bar{\text{C}}_\text{Hu}$ limits}
Thus, the presented ttZ $\bar{\text{C}}_\text{Hu}$ limits of Ref.~\cite{Sirunyan:2017uzs} convert to \texttt{dim6top} limits as
\begin{align}
\text{C}_{\phi\text{t}}^\text{Top 17-005$\rightarrow$ dim6top}(36.1~\text{fb}^{-1}, 95\%, \text{x-sec only}):\, [-20.2, +4.0]
\end{align}

%where $C_{\phi\text{t}}$ is dimensionless, however typically it is given as $C_{\phi\text{t}}\,v^2/\Lambda^2$, in units of [(TeV/$\Lambda$)$^2$]. $C_\text{Hu}$ has units of [TeV$^{2}$], which often, as e.g. in Ref.~\cite{Sirunyan:2017uzs}, is converted to the dimensionless variable $\bar{C}_\text{Hu}$, with
%\begin{align}
%\bar{C}_\text{Hu} = \frac{C_\text{Hu}}{v^2}
%\end{align}
%where limits are given in the form $\bar{C}_\text{Hu}/\Lambda^2$ in units of $[\text{TeV}^{-2}]$ or on $\bar{C}_\text{Hu}$ in units of $[(\Lambda/\text{TeV})^{-2}]$. The limits in this notation are then consistant with $C_{\phi\text{t}}$~[($\Lambda$/TeV)$^2$], however with the conversion factor
%\begin{align}
%C_{\phi\text{t}}\,[(\Lambda/\text{TeV})^2] &= 2\,\bar{C}_\text{Hu}\,[(\Lambda/\text{TeV})^{-2}]
%\end{align}
%\noindent

\clearpage

\bibliographystyle{unsrt}
\bibliography{biblio}
\end{document}
